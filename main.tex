\documentclass[aspectratio=169,xcolor=table]{beamer}
\usepackage{algorithm}
\usepackage{algpseudocode}
\usepackage[utf8]{inputenc}
\usepackage[T1]{fontenc}
\usepackage{lipsum, lmodern}
\usepackage{csquotes}
\usepackage{xcolor}
\usepackage[portuguese]{babel}

% ------------------------------------------------
% Tema do Beamer (exemplo de um tema customizado)
% ------------------------------------------------
\usetheme{DCC} % <-- Ajuste conforme seu tema ou estilo

\graphicspath{{imgs/}{./imgs/}}

\author[Magalhães, Felipe]{%
  \textbf{Antoniel Magalhães} \\
  \textbf{Luis Felipe}
}
\title{Simulação de ondas e oceano}
\institute{Universidade Federal da Bahia \\ Instituto de Computação}
\date{\today}

\begin{document}

%-------------------------------------------------
%  SLIDE DE TÍTULO
%-------------------------------------------------
\begin{frame}[plain,noframenumbering]
    \titlepage
\end{frame}

%-------------------------------------------------
%  SLIDE DE AGENDA
%-------------------------------------------------
\begin{frame}{Agenda}
    \tableofcontents
\end{frame}

\setlength{\parskip}{1em} % Adjust the space between paragraphs

%=================================================
\section{Introdução}
%=================================================
\begin{frame}{Introdução}
    \begin{itemize}
        \item A simulação de ondas e oceano é uma área de estudo que combina física, matemática e computação para modelar o comportamento das ondas no mar.
        \item Este campo é crucial para aplicações em engenharia costeira, previsão do tempo e estudos ambientais.
    \end{itemize}
\end{frame}

%=================================================
\section{Teoria Linear ou Teoria de Onda de Pequena Amplitude}
%=================================================
\begin{frame}{Teoria Linear ou Teoria de Onda de Pequena Amplitude}
    \begin{itemize}
        \item A abordagem mais elementar da teoria de ondas superficiais de gravidade é conhecida como teoria linear ou teoria de pequena amplitude.
        \item Desenvolvida por Airy em 1845, esta teoria considera em seus cálculos o caso mais simples da propagação do campo de ondas na ausência de qualquer forçante.
        \item Apesar das simplificações impostas, esta teoria tem uma extensa gama de aplicações \cite{meirelles2007modelagem}.
    \end{itemize}
\end{frame}

\begin{frame}{Assumptions of Linear Theory}
    \begin{itemize}
        \item A teoria linear assume que: o fluido é homogêneo, incompressível (densidade constante) e irrotacional, permitindo a existência do potencial de velocidade.
        \item A tensão superficial é desprezada.
        \item A pressão na superfície livre é uniforme e constante.
        \item O fluido é invíscido.
        \item O fundo é um limite plano, horizontal, fixo e impermeável.
        \item A amplitude da onda é constante e pequena em relação ao comprimento e à profundidade.
    \end{itemize}
\end{frame}

\begin{frame}{Equação de Laplace}
    \begin{itemize}
        \item Em face das suposições iniciais supracitadas, torna-se possível desenvolver as formulações da teoria linear a partir da solução da equação de Laplace.
        \item Esta equação é obtida reescrevendo a equação da continuidade no plano x,z em função do potencial de velocidade.
    \end{itemize}
\end{frame}

%=================================================
\section{Simulação do Empinamento}
%=================================================
\begin{frame}{Simulação do Empinamento}
    \begin{itemize}
        \item O empinamento das ondas é um fenômeno importante na dinâmica oceânica.
        \item A simulação deste processo ajuda a entender como as ondas interagem com estruturas costeiras e como a energia das ondas é dissipada.
    \end{itemize}
\end{frame}

%=================================================
\section{Computação Gráfica na Simulação de Ondas}
%=================================================
\begin{frame}{Computação Gráfica na Simulação de Ondas}
    \begin{itemize}
        \item A computação gráfica desempenha um papel vital na visualização das simulações de ondas.
        \item Técnicas avançadas permitem a criação de modelos visuais realistas que ajudam na análise e interpretação dos dados simulados.
    \end{itemize}
\end{frame}

%=================================================
\section{Referências}
%=================================================
\begin{frame}[allowframebreaks]{Referências}
    \bibliographystyle{plain}
    \bibliography{Bibliografia}
\end{frame}

\end{document}
