\documentclass[aspectratio=169,xcolor=table]{beamer}
\usepackage{algorithm}
\usepackage{algpseudocode}
\usepackage[utf8]{inputenc}
\usepackage[T1]{fontenc}
\usepackage{lipsum, lmodern}
\usepackage{csquotes}
\usepackage{xcolor}
\usepackage[portuguese]{babel}
\usepackage{amsmath}
\usepackage{physics}

% ------------------------------------------------
% Tema do Beamer (exemplo de um tema customizado)
% ------------------------------------------------
\usetheme{DCC} % <-- Ajuste conforme seu tema ou estilo

\graphicspath{{imgs/}{./imgs/}}

\author[Magalhães, Felipe]{%
  \textbf{Antoniel Magalhães} \\
  \textbf{Luis Felipe}
}
\title{Simulação de ondas e oceano}
\institute{Universidade Federal da Bahia \\ Instituto de Computação}
\date{\today}

\begin{document}

%-------------------------------------------------
%  SLIDE DE TÍTULO
%-------------------------------------------------
\begin{frame}[plain,noframenumbering]
    \titlepage
\end{frame}

%-------------------------------------------------
%  SLIDE DE AGENDA
%-------------------------------------------------
\begin{frame}{Agenda}
    \tableofcontents
\end{frame}

\setlength{\parskip}{1em} % Adjust the space between paragraphs

%=================================================
\section{Introdução}
%=================================================
\begin{frame}{Introdução}
    \begin{itemize}
        \item A simulação de ondas e oceano é uma área de estudo que combina física, matemática e computação para modelar o comportamento das ondas no mar.
        \item Este campo é crucial para aplicações em engenharia costeira, previsão do tempo e estudos ambientais.
    \end{itemize}
\end{frame}

%=================================================
\section{Teoria Linear ou Teoria de Onda de Pequena Amplitude}
%=================================================
\begin{frame}{Teoria Linear ou Teoria de Onda de Pequena Amplitude}
    \begin{itemize}
        \item A abordagem mais elementar da teoria de ondas superficiais de gravidade é conhecida como teoria linear ou teoria de pequena amplitude.
        \item Desenvolvida por Airy em 1845, esta teoria considera em seus cálculos o caso mais simples da propagação do campo de ondas na ausência de qualquer forçante.
        \item Apesar das simplificações impostas, esta teoria tem uma extensa gama de aplicações \cite{meirelles2007modelagem}.
    \end{itemize}
\end{frame}

\begin{frame}{Pressupostos da Teoria Linear}
    \begin{itemize}
        \item A teoria linear assume que: o fluido é homogêneo, incompressível (densidade constante) e irrotacional, permitindo a existência do potencial de velocidade.
        \item A tensão superficial é desprezada.
        \item A pressão na superfície livre é uniforme e constante.
        \item O fluido é invíscido.
        \item O fundo é um limite plano, horizontal, fixo e impermeável.
        \item A amplitude da onda é constante e pequena em relação ao comprimento e à profundidade.
    \end{itemize}
\end{frame}

\begin{frame}{Equação de Laplace}
    \begin{itemize}
        \item A equação de Laplace para o potencial de velocidade $\phi(x,z,t)$ em duas dimensões é:
        \[\nabla^2\phi = \frac{\partial^2\phi}{\partial x^2} + \frac{\partial^2\phi}{\partial z^2} = 0\]
        \item As condições de contorno são:
        \begin{itemize}
            \item Cinemática da superfície livre: $\frac{\partial \eta}{\partial t} = \frac{\partial \phi}{\partial z}$ em $z = \eta(x,t)$
            \item Dinâmica da superfície livre: $\frac{\partial \phi}{\partial t} + g\eta = 0$ em $z = \eta(x,t)$
            \item Condição de fundo: $\frac{\partial \phi}{\partial z} = 0$ em $z = -h$
        \end{itemize}
    \end{itemize}
\end{frame}

\begin{frame}{Solução da Equação de Laplace}
    \begin{itemize}
        \item A solução para o potencial de velocidade é:
        \[\phi(x,z,t) = \frac{gH}{2\omega}\frac{\cosh[k(h+z)]}{\cosh(kh)}\sin(kx-\omega t)\]
        \item Onde:
        \begin{itemize}
            \item $H$ é a altura da onda
            \item $\omega$ é a frequência angular
            \item $k$ é o número de onda
            \item $h$ é a profundidade
        \end{itemize}
    \end{itemize}
\end{frame}

\begin{frame}{Aspectos Computacionais}
    \begin{itemize}
        \item Na computação gráfica, a superfície da água é frequentemente representada como uma malha de vértices
        \item A elevação da superfície $\eta(x,t)$ é dada por:
        \[\eta(x,t) = \frac{H}{2}\cos(kx-\omega t)\]
        \item A relação de dispersão conecta frequência e número de onda:
        \[\omega^2 = gk\tanh(kh)\]
    \end{itemize}
\end{frame}

\begin{frame}{Técnicas de Renderização}
    \begin{itemize}
        \item Métodos de renderização em tempo real:
        \begin{itemize}
            \item Normal mapping para detalhes da superfície
            \item Fresnel effect para reflexão/refração
            \item Caustics para efeitos de luz subaquática
        \end{itemize}
        \item A velocidade das partículas é dada por:
        \[\vec{v} = \nabla\phi = \begin{pmatrix}
            \frac{\partial \phi}{\partial x} \\
            \frac{\partial \phi}{\partial z}
        \end{pmatrix}\]
    \end{itemize}
\end{frame}

\begin{frame}{Otimizações e Performance}
    \begin{itemize}
        \item Técnicas de Level of Detail (LOD):
        \begin{itemize}
            \item Tessellation adaptativa baseada na distância da câmera
            \item Redução de vértices em áreas distantes
        \end{itemize}
        \item Fast Fourier Transform (FFT) para síntese de ondas:
        \[\eta(x,y,t) = \sum_{i=1}^N \sum_{j=1}^N A_{ij}\cos(\vec{k_{ij}}\cdot\vec{x} - \omega_{ij}t + \phi_{ij})\]
    \end{itemize}
\end{frame}

%=================================================
\section{Simulação do Empinamento}
%=================================================
\begin{frame}{Simulação do Empinamento}
    \begin{itemize}
        \item O empinamento das ondas é um fenômeno importante na dinâmica oceânica.
        \item A simulação deste processo ajuda a entender como as ondas interagem com estruturas costeiras e como a energia das ondas é dissipada.
    \end{itemize}
\end{frame}

%=================================================
\section{Computação Gráfica na Simulação de Ondas}
%=================================================
\begin{frame}{Computação Gráfica na Simulação de Ondas}
    \begin{itemize}
        \item A computação gráfica desempenha um papel vital na visualização das simulações de ondas.
        \item Técnicas avançadas permitem a criação de modelos visuais realistas que ajudam na análise e interpretação dos dados simulados.
    \end{itemize}
\end{frame}

%=================================================
\section{Referências}
%=================================================
\begin{frame}[allowframebreaks]{Referências}
    \bibliographystyle{plain}
    \bibliography{Bibliografia}
\end{frame}

\end{document}
